\documentclass{article}
%% INCLUDES
\usepackage{amsmath}
\usepackage{marginnote}

%% MY STYLES
\newcommand{\sidenote}[1]{\textit{#1}\marginnote{\textbf{#1}}}
\reversemarginpar

%% CONSTRUCTOR ARGUMENTS
\title{\textbf{TDT1337 - Artificial Beings}}
\author{Fredrik Chrislock \& Sigurd Murad}
\date{~} 

%% INIT
\begin{document}
	\maketitle
	\section{Introduction}
	In the later years there has been a significant increase in interest around artificial intelligence and how it can be used to solve tasks that before was impossible by machines. This spike in interest is because of the recent development in computer science, and how it has enabled machine learning to be a useful tool. 
	
	It is consensus among researchers that we still are a long way from achieving general AI, or more specifically, human level intelligence. 
	
	TDT1337 - Artificial Beings is a project aiming to apply the latest technology within artificial intelligence and machine learning to simulate an artificial being, living within a virtual world. The motivation for doing this is that through modeling, we can gradually improve the beings performance trough incremental development and genetic development.
	
	\subsection{The setup}
	The artificial being is living in a virtual world, confined within a computer simulation. Now, we will define the layout of this simulation.\\
	\textbf{The world}\\
	The virtual world is at the most basic level a 3D matrix instance, where each element is a point in virtual space. Any object must exist within this matrix. There is a single module governing and maintaining this world, called the \sidenote{world engine}. Basically the world engine does three things:
	\begin{enumerate}
		\item Ensuring the laws of physics are held, including basic axioms such as "two objects of matter cannot take up the same geometric point"
		\item Rendering the simulation such that it can be displayed to us.
		\item Provide gateways for any algorithm that want to influence the world
	\end{enumerate}
	\textbf{Passive objects}\\
	This is everything in the world that just exist, e.g. rocks, water and air. These object are described by basic properties such as geometry, mass, temperature, location and speed.\\
	\textbf{Environmental agents}\\
	These are algorithms working outside the world, giving "life" to the world  trough applying energy and forces to any object. These are responsible for weather, fire, radiation, light and all other things we would put under the umbrella-term "mother nature ". \\
	\textbf{Active objects}\\
	This is where it becomes interesting. Active objects are capable of information processing and integration, which basically make them any "living" thing in the world. An active object is a vessel driven by a soul, or more practically a object that is connected to a outside agent (the soul) through a dedicated gateway. It might be easier to visualize it as this: When playing a video game, the active object is the character you are playing, the controller and display is the gateway, and you are the agent.
	
	\subsection{Agents}
	When developing artificial beings, we are actually talking about programming agents, giving them new features and measuring their ability to survive in the world. 
	
	\section{The first task}
	The first experiment is simple and involves no programming of artificial intelligence or machine learning. It will work as a template for how any experiment in this project will be executed.
	\subsection{Goal}
	The first thing in any task is to define the goal, -what we aim to accomplish. For the first task, the goal is to make a simple artificial being living in a very small world, and have it live and die. 
	\subsection{Setup}
	The world is a surface on which the artificial being can stand, and it cold and exposed to radiation. The being is simple cell trying not to deteriorate. It is initialized with an amount of stored energy, a membrane, and a mechanism for fixing the membrane if broken. In the core of the cell lies the \sidenote{soul}. The soul is extremely fragile and dies from almost anything. In this task two things kills it: 
	\begin{itemize}
		\item Being too cold
		\item Being directly exposed to radiation
	\end{itemize}
	The agent running the being have the following gateway description:
	\begin{itemize}
		\item It can ping the soul connection (test connection). If no answer, the being has died.
		\item It can measure the temperature in the cell
		\item It can check whether the membrane status (Intact/damaged). A third option could be "broken" however then the soul is directly exposed to radiation and the soul connection is severed, so no information can be sendt to the agent.
		\item It can "burn energy unit" to keep the cell warm
		\item It can use the command "repair", this costs a unit of energy.
	\end{itemize}
	\subsection{Agent model}
	The agent is an object (as in object-oriented programming) that in a way simulate the abstract part of the brain. It's in here the neural networks go, and any high reasoning algorithm. Non-ML algorithms can be thought of as the brain "skeleton", or any automatic features of the brain, such as REM cycle maintenance, keeping the heart beating, storing memories and the like. In this first task the brain is only algorithms performing tasks to keep the cell alive as long as possible. 
\end{document}